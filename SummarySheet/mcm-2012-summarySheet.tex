% !TEX TS-program = pdflatex
% !TEX encoding = UTF-8 Unicode

% This is a simple template for a LaTeX document using the "article" class.
% See "book", "report", "letter" for other types of document.

\documentclass[11pt]{article} % use larger type; default would be 10pt

\usepackage[utf8]{inputenc} % set input encoding (not needed with XeLaTeX)

%%% Examples of Article customizations
% These packages are optional, depending whether you want the features they provide.
% See the LaTeX Companion or other references for full information.

%%% PAGE DIMENSIONS
\usepackage{geometry} % to change the page dimensions
\geometry{a4paper} % or letterpaper (US) or a5paper or....
% \geometry{margins=2in} % for example, change the margins to 2 inches all round
% \geometry{landscape} % set up the page for landscape
%   read geometry.pdf for detailed page layout information

\usepackage{graphicx} % support the \includegraphics command and options

% \usepackage[parfill]{parskip} % Activate to begin paragraphs with an empty line rather than an indent

%%% PACKAGES
\usepackage{booktabs} % for much better looking tables
\usepackage{array} % for better arrays (eg matrices) in maths
\usepackage{verbatim} % adds environment for commenting out blocks of text & for better verbatim
\usepackage{subfig} % make it possible to include more than one captioned figure/table in a single float
\usepackage{amsmath, amssymb, amsthm, lastpage}
% These packages are all incorporated in the memoir class to one degree or another...

%%% HEADERS & FOOTERS
\usepackage{fancyhdr} % This should be set AFTER setting up the page geometry
\pagestyle{fancy} % options: empty , plain , fancy
\renewcommand{\headrulewidth}{0pt} % customise the layout...
\lhead{Team \# 16677}\chead{}\rhead{Page \thepage\ of \pageref{LastPage}}
\lfoot{}\cfoot{\thepage}\rfoot{}

%%% SECTION TITLE APPEARANCE
%\usepackage{sectsty}
%\allsectionsfont{\sffamily\mdseries\upshape} % (See the fntguide.pdf for font help)
% (This matches ConTeXt defaults)

%%% ToC (table of contents) APPEARANCE
%\usepackage[nottoc,notlof,notlot]{tocbibind} % Put the bibliography in the ToC
%\usepackage[titles,subfigure]{tocloft} % Alter the style of the Table of Contents
%\renewcommand{\cftsecfont}{\rmfamily\mdseries\upshape}
%\renewcommand{\cftsecpagefont}{\rmfamily\mdseries\upshape} % No bold!
\usepackage{setspace}
\onehalfspacing

%%% END Article customizations

%%% The "real" document content comes below...
%\date{}

\begin{document}
We approach the challenge of scheduling river rafting trips throughout a 6
month season as an assignment problem.  That is, for each day we seek to
assign each travelling group (boats travelling together) to a campsite
destination such that no two groups are assigned to reach the same site on
the same day.  It is also desirable that groups should encounter each other
as little as possible, epitomizing the outdoor experience, and that we
should be able to assign as many groups as possible throughout the season.
Based on research and necessity, we assume that there are approximately 100
camping sites; groups are capable of passing one another at any point; the
river does not increase travel speed at any time in the trip; a group stays
as a unit throughout the entire length of the river; and the system is
deterministic, with travelling groups allowing us to dictate their exact
schedule.  While there are crafts that travel only 4 or
only 8 mph, we relax the problem by introducing variable times on the river
and $tolerance$ (possible deviation on number of campsites passed each
day) for all groups.  These relaxations lead to a significantly higher
number of possible assignments.

The program we wrote utilizes constrained but randomly generated travel
groups to conduct simulations in testing the solution algorithm.  Our
solution approaches scheduling as a constraint satisfaction problem (CSP).
This class of problem seeks to assign values to variables: campsites to
travelling groups in our case.  This approach allows us to explore solutions
with preference for group satisfaction (proximity of actual schedule to
ideal trip) or any other number of variables.  Additionally, we shaped the
problem such that each day is a CSP in and of itself, creating a dynamic CSP
where each day depends on the last.  Our model found that not all numbers of
travelling groups are possible under ``standard'' conditions, but we expected
this and still managed to find promising results.

In analyzing the model, we focused primarily on the size of groups possible
in a given season.  Our findings suggested that a successful schedule is nearly
always found for 200 groups in a season, with tolerance 2.  It is certainly
possible to schedule many more groups in a season, but there is a diminishing
probability that a schedule will be found.  At about 300 travelling groups,
the scheduling success rate begins to drop, sharply, below 91\% until
around 600 groups the success rate begins to level off, asymptotically, at
5\%.  We note that this drop in success roughly follows an inverse of the
possible number of schedules given group size (an exponential growth) and
that satisfaction is relatively equivalent to success rate.  Lastly, our
model found that utilizing more motorboats than oar-powered rafts led to a
greater number of successes.  Further discussion and analysis within the
paper addresses the greater implications of these results along with possible
extensions to the model.

\end{document}
