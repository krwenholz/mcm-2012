% !TEX TS-program = pdflatex
% !TEX encoding = UTF-8 Unicode

% This is a simple template for a LaTeX document using the "article" class.
% See "book", "report", "letter" for other types of document.

\documentclass[11pt]{article} % use larger type; default would be 10pt

\usepackage[utf8]{inputenc} % set input encoding (not needed with XeLaTeX)

%%% Examples of Article customizations
% These packages are optional, depending whether you want the features they provide.
% See the LaTeX Companion or other references for full information.

%%% PAGE DIMENSIONS
\usepackage{geometry} % to change the page dimensions
\geometry{a4paper} % or letterpaper (US) or a5paper or....
% \geometry{margins=2in} % for example, change the margins to 2 inches all round
% \geometry{landscape} % set up the page for landscape
%   read geometry.pdf for detailed page layout information

\usepackage{graphicx} % support the \includegraphics command and options

% \usepackage[parfill]{parskip} % Activate to begin paragraphs with an empty line rather than an indent

%%% PACKAGES
\usepackage{booktabs} % for much better looking tables
\usepackage{array} % for better arrays (eg matrices) in maths
\usepackage{verbatim} % adds environment for commenting out blocks of text & for better verbatim
\usepackage{subfig} % make it possible to include more than one captioned figure/table in a single float
\usepackage{amsmath, amssymb, amsthm, lastpage}
% These packages are all incorporated in the memoir class to one degree or another...

%%% HEADERS & FOOTERS
\usepackage{fancyhdr} % This should be set AFTER setting up the page geometry
\pagestyle{fancy} % options: empty , plain , fancy
\renewcommand{\headrulewidth}{0pt} % customise the layout...
\lhead{Team \# 16677}\chead{}\rhead{Page \thepage\ of \pageref{LastPage}}
\lfoot{}\cfoot{}\rfoot{}

%%% SECTION TITLE APPEARANCE
%\usepackage{sectsty}
%\allsectionsfont{\sffamily\mdseries\upshape} % (See the fntguide.pdf for font help)
% (This matches ConTeXt defaults)

%%% ToC (table of contents) APPEARANCE
%\usepackage[nottoc,notlof,notlot]{tocbibind} % Put the bibliography in the ToC
%\usepackage[titles,subfigure]{tocloft} % Alter the style of the Table of Contents
%\renewcommand{\cftsecfont}{\rmfamily\mdseries\upshape}
%\renewcommand{\cftsecpagefont}{\rmfamily\mdseries\upshape} % No bold!
\usepackage{setspace}
\onehalfspacing

%%% END Article customizations

%%% The "real" document content comes below...
%\date{}

\begin{document}
We approach the challenge of scheduling river rafting trips as an
assignment problem. That is, for each day of the six month season we seek to
assign each visitor a campsite.
No two visitors are allowed the same site on
the same day.  While it is desirable that groups encounter each other
as little as possible, epitomizing the outdoor experience, this is second to
assigning as many groups as possible.
Based on research and necessity, we assume there are 128
camp sites; groups are capable of passing one another at any point;
travel speed is constant; a group stays
as a unit throughout the entire length of the river; and, excluding
departure date, visitors allow us to dictate their exact
schedule.  While there are crafts that travel only 4 or
only 8 mph, we relax the problem by introducing variable times on the river
each day,
and by introducing a $tolerance$ on the number of campsites passed each day.
These relaxations allow a larger number of possible assignments.

The program we wrote utilizes reasonable randomly generated travel
groups to conduct simulations in testing the solution algorithm.  Our
solution approaches scheduling as a constraint satisfaction problem (CSP).
This class of problem seeks to assign values to variables.  In our case,
this means assigning campsites to visitors once per day.
This allows us to explore solutions
with preference for visitor satisfaction or any other number of variables.
Additionally, we shaped the
problem such that the season is a CSP in and of itself.  This creates a dynamic
CSP where each day depends upon the last.  Our model suggests a limit to the
number of visitors possible under ``standard'' conditions.

In analyzing the model, we focused primarily on the number of visitors possible
in a given season.  Our findings suggested that a successful schedule is nearly
always found with 200 visitors per season and a 2-site tolerance.  While it is
possible to schedule many more groups per season, there is a diminishing
probability a valid schedule will be found.  At 300 visitors per season,
the probability of successfully scheduling a season begins to drop sharply.
It drops below 91\% and continues decreasing
until reaching an asymptote of 5\% at 600 visitors per
season. As the solution space grows exponentially, the success rate decreases
exponentially.  We also considered satisfaction and found it
decreased with success rate.  Lastly, our
model found that utilizing more motorboats than oar-powered rafts led to a
greater number of successes.  The paper addresses further implications of
these results along with possible extensions to the model.

\end{document}
