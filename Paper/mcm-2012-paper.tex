% !TEX TS-program = pdflatex
% !TEX encoding = UTF-8 Unicode

% This is a simple template for a LaTeX document using the "article" class.
% See "book", "report", "letter" for other types of document.

\documentclass[11pt]{article} % use larger type; default would be 10pt

\usepackage[utf8]{inputenc} % set input encoding (not needed with XeLaTeX)

%%% Examples of Article customizations
% These packages are optional, depending whether you want the features they provide.
% See the LaTeX Companion or other references for full information.

%%% PAGE DIMENSIONS
\usepackage{geometry} % to change the page dimensions
\geometry{a4paper} % or letterpaper (US) or a5paper or....
% \geometry{margins=2in} % for example, change the margins to 2 inches all round
% \geometry{landscape} % set up the page for landscape
%   read geometry.pdf for detailed page layout information

%%% Line Spacing
\usepackage{setspace}
\onehalfspacing

\usepackage{graphicx} % support the \includegraphics command and options

% \usepackage[parfill]{parskip} % Activate to begin paragraphs with an empty line rather than an indent

%%% PACKAGES
\usepackage{booktabs} % for much better looking tables
\usepackage{array} % for better arrays (eg matrices) in maths
\usepackage{verbatim} % adds environment for commenting out blocks of text & for better verbatim
\usepackage{subfig} % make it possible to include more than one captioned figure/table in a single float
\usepackage{amsmath, amssymb, amsthm, lastpage}
% These packages are all incorporated in the memoir class to one degree or another...

%%% HEADERS & FOOTERS
\usepackage{fancyhdr} % This should be set AFTER setting up the page geometry
\pagestyle{fancy} % options: empty , plain , fancy
\renewcommand{\headrulewidth}{0pt} % customise the layout...
\lhead{Team \# 16677}\chead{}\rhead{Page \thepage\ of \pageref{LastPage}}
\lfoot{}\cfoot{\thepage}\rfoot{}

%%% SECTION TITLE APPEARANCE
%\usepackage{sectsty}
%\allsectionsfont{\sffamily\mdseries\upshape} % (See the fntguide.pdf for font help)
% (This matches ConTeXt defaults)

%%% ToC (table of contents) APPEARANCE
%\usepackage[nottoc,notlof,notlot]{tocbibind} % Put the bibliography in the ToC
%\usepackage[titles,subfigure]{tocloft} % Alter the style of the Table of Contents
%\renewcommand{\cftsecfont}{\rmfamily\mdseries\upshape}
%\renewcommand{\cftsecpagefont}{\rmfamily\mdseries\upshape} % No bold!

%%% END Article customizations

%%% The "real" document content comes below...
%\date{}

\begin{document}
\begin{titlepage}
    \vspace*{\fill}
    \begin{center}
      \Huge{Rapid Scheduling: A Constraint Satisfaction
        Approach to River Usage}\\[0.5cm]
      \Large{Team: 16677}\\[0.4cm]
      \today
    \end{center}
    \vspace*{\fill}
  \end{titlepage}
\newpage
\vspace*{\fill}
\tableofcontents
\vspace*{\fill}
\newpage

\section{Preliminaries}
\label{sec:prelims}
%Possibly include some comparisons of our model to existing research if any.

\subsection{Introduction}
\label{sec:intro}
Floating down a river is easy enough, but imagine scheduling an entire
season of rafting trips for hundreds of people.  A \textit{scheduling problem}
occurs when there are start and end times and places for a group of people
or objects. The schedules of the group must be arranged such that all
participants start and stop as desired. Between the start and stop, they
must interact in a certain way with each other and their surroundings.

We consider the scheduling problem for a river with visitors who desire to
raft down it. The goal is to schedule as many visitors as possible.
Visitors, however, are seeking a relaxing and authentic experience, and
will become unsatisfied if they do not receive one. While scheduling
unsatisfied visitors is allowed, we seek to schedule them in such a way
that as many visitors are as satisfied as possible.


\subsection{Definitions and Terminology}
\label{sec:defs}
% CSP a group of generally NP-hard problems.
% travel group
% itinerary
% NP-hard
% big O
% river length = L
% days in the season = S
% happiness as satisfaction: related to encounters and actual trip speed
% tolerance: brief description plus see section blah
% carrying capacity
\begin{itemize}
\item Big O: (of an algorithm) Also "time complexity." A function relating the
time it takes to solve a problem to the inputs of the problem. For instance,
a function whose big O is O(x) computes in linear time.

\item Carrying Capacity: The amount of visitors a river can see in a single
year, while maintaining the likelihood of successfully scheduling all
visitors and keeping them satisfied.

\item Constraint Satisfaction Problem (CSP): A type of problem solved by
assigning values to variables. These values must be within specified domains.
The problem is solved when all variables have values within the specified
domains -- eg, they have values such that all constraints are satisfied.

\item Group: See "visitor."

\item Itinerary: The ordered list of campsites a visitor will stay at each
night.

\item Non-deterministic Polynomial-Time Hard Problem (NP-Hard): A class of
problems whose solutions, if they exist, behave unpredictably, and are
provable in polynomial time -- see "Big O." CSPs
are a type of NP-hard problem.

\item Visitor: Also "group" or "travel group." A person or a group of people
seeking to travel the Big Long River by raft. Their raft could be powered
by oar, or it could be motorized. Visitors are very irritable and will
become unhappy if they see other visitors, don't get their desired water
time each day, or don't get scheduled.

\item Satisfaction: (of a visitor) Visitors seeking to raft the Big Long River
start with a satisfaction of 100. This number will drop each time the visitor
sees another visitor, or doesn't get their desired water time (too much or
too little of it) each day. It is possible for a visitor's satisfaction to
drop to zero by these two factors alone. It will be zero if the visitor
can't be scheduled.

\item Season: A six month season is assumed to have 180 days; that is,
six months times an average of 30 days per month.

\item Tolerance: The amount of campsites a visitor's travel plan can be
altered by. For instance, a tolerance of 1 would allow a visitor desiring
to move four campsites each day to move between three and five campsites
each day. This is done in order to accommodate for other visitors who may
be in the desired spot.

\item Water Time: Visitors specify how much time they want to spend on the
water each day. This value is the same for each day of the trip. A visitors'
satisfaction rating will drop if they are forced to spend
more or less than this amount of time on the water.

\item X: The current number of trips taken by visitors down the Big Long River
each year.

\item Y: The number of campsites on the Big Long River. Unless otherwise stated,
this was assumed to be 128, as provided by literature\cite{NumCampsites}.
\end{itemize}

\subsection{Assumptions and Simplifications}
\label{sec:assumptions}
% season length
% days traveled opposed to nights
% groups
% encounters
% happiness calculation as satisfaction (show equations)
\begin{itemize}
\item Season Length: A six month season is assumed to have 180 days; that is,
six months times an average of 30 days per month.  Let it also be noted that
this season is consistent (i.e. there exist no peak visitation months or
months of higher flow).

\item Trips ranging from "6 to 18 nights of camping on the river, start to
finish" are assumed to allow
between 7 and 19 days of travel on the river, start to finish.

\item Visitors or groups of visitors travel at a constant speed throughout the
trip, which is defined only by the type of boat they use. Oar-powered
rubber rafts travel at 4 mph, and motorized boats travel at 8 mph.

\item Visitors do not separate into more than one group or combine with other
groups. Therefore, any group of visitors is scheduled the same as a single
visitor.

\item "Contact with other groups of boats on the river" is only counted if
one group passes another group, or is passed by another group.

\item Groups are capable of passing one another at any point (i.e. passing is
not restricted by the width of the river or anything else at anytime).

\item The satisfaction of a visitor is affected exclusively by the amount of
contact with other groups that visitor has, how much water time they have
compared to their desired amount, and whether or not they
are actually scheduled.

\item Visitors specify only their starting date, boat type, and water time,
and only valid values for these parameters. Therefore: They may not start
before the start of the season, or so close to the end of the season it
would be impossible for them to finish; they may not have a water time
such that their trip takes shorter than 7 or longer than 19 days; they may
only choose an oar-powered rubber raft or a motorized boat.

\item Visitors are willing to follow the schedule given to them exactly:
no backtracking, skipping, or slowing down.
\end{itemize}


\section{The Solution}
\label{sec:model-details}
Our model is structured as a constraint satisfaction problem solver.
The general definition
for such a problem is that - given variables ($X$), values ($D$),
and constraints ($C$) - find an assignment of value(s) in $D$
for each $X$ such that all constraints in $C$ are satisfied.  In general
then, a CSP may be defined as a set $(X,D,C)$.
Good examples of this problem set are 2-SAT and coloring problems\cite{DynamicCSP}.

We view the problem of scheduling rafting trips as a progressive and dynamic
CSP: each day being a constraint satisfaction problem dependent upon the
solution to the previous day's problem.  By limiting the values any variable
may take on and applying efficient algorithms, the CSP becomes tractable and
we may successfully schedule river trips without the need of unreasonable
computing power.

\subsection{Constraint Satisfaction}
\label{sec:csp}
%variables, values, constraints (rules from problem statement)
%travel groups as variables
%campsites as values
%constraints

In general, our approach is to first consider that there are exponentially
many combinations of  visitors to campsites during their trip.  Many of
these combinations, however, are invalid given the constraint of no two
 visitors being allowed to stay at the same campsite.  This is perhaps
the greatest limiting factor in the problem as it applies to all campers
passing through the same territory on a given day.  This is an $n-ary$
constraint: the constraint relating $n$ variables.  Such constraints are
computationally challenging for many problems, but our problem space is limited
enough that the computations are relatively painless ($O(n^2)$ with $n$ as
number of  visitors) to pair all campers and detect collisions at any
campsite.  There were few other constraints involved in solving the problem
as we were mostly interested in solutions to our schedule.

As we have been discussing, each  visitor was considered to be a variable
in the CSP and campsites were considered the values.  The domain of these
values (for each  visitor) varied day to day depending upon the distance
each visitor was capable of travelling.  This travel distance was derived from
a visitor's average speed of 4 or 8 mph, $v$,  multiplied with a uniformly distributed
random variable for desired daily time spent on the water, $t$.  This variable is such
that the travel distance per day would allow the visitor to complete the trip
within the allotted time of 7-19 days. The following few equations explicitly
describe these relationships:
$$v*t=OptimalDayDistance$$
$$v*t*7\geq L$$
So in generating any  visitor, we ensure the visitor is reasonably
parameterized for finishing a trip on time.  Each visitor is also randomly
assigned, uniformly distributed, a day for departure (dDay) such that
$$v*t*7+dDay\leq S$$
That is, a visitor's maximum travel speed will allow them to finish before
the close of the season.

Visitors are generated en masse using a varitey of uniformly distributed
variables.  It is possible to load in a file containing visitor information
so that the program will work on problems of corporeal, rather than testing
and verification, import.  Once visitors are generated and basic parameters
are set (discussed in \textbf{Section \ref{sec:high-params}}) the program begins
attempting to produce a solution through our CSP solver.

\subsubsection{Backtracking Search}
In order to solve a CSP, one must navigate the vast multitude of combinations
for variables and values. We utilize a backtracking search algorithm to
assign campsites to our  visitors on a daily basis\cite{AI-Intro}.
On any given day
$d$, the solver looks to see which visitors are set to depart on $d$ and
subsequently adds them to a list, $toDepart$.  All visitors still on the
river from the previous day are added to $toDepart$, creating a list of all
 visitors requiring some movement for day $d$.  In order to determine
appropriate movements, the visitors are assigned campsites (analagous to
distance traveled for the day) one-by-one.  Following
a pairing, the full day's assignment is checked to see if this pairing is
consistent with current progress.  If the pairing doesn't violate any other
existing pairings, it is added to the day's list of assignments.  We then
proceed to the next  visitor.

Encountering failed pairings causes the algorithm to attempt another pairing
until success is achieved or failure must be returned.  If failure occurs,
the algorithm returns to the most recent pairing, undoes this pairing, and
then attempts a new pairing.  The entire process is structured as a
depth-first-search. When a day's assignment is complete, the algorithm creates
a new CSP for the next day and executes again, deepening the recursing into
another day.  \textbf{Figure~\ref{fig:searchExample}} highlights the major steps
involved in this process and provides a simple example.  The example is
simplified so that only two visitors are being assigned a possibility of two
campsites.  The algorithm begins by assigning site $1$ to visitor $A$, denoted
by the diamond below and to the left of the tree-top.  Then assigning
site $1$ to $B$ is tried but found to fail, so the algorithm assigns site
$2$ to $B$ (rightmost circle).  At this point, a the day would ``end'' and
the algorithm would begin attempting to schedule the new day.  In the event
of a failure on a new day, the algorithm continues jumping up a level, as
seen in the example, even to the extent of needing to reschedule a previous
day.

\begin{figure}[h]
  \centering
  \includegraphics[scale=0.5]{imgs/searchExample.png}
  \caption{An example of the backtracking algorithm's attempt to schedule
     visitors $A$ and $B$ with campsites $1$ and $2$.}
  \label{fig:searchExample}
\end{figure}

Our solver is optimized to attempt maximal satisfaction of the travelers.
That is, the possible campsites a visitor may travel to are presented to the
solver in order of ideal travel distance.  While not every visitor is guaranteed
to always travel their desired distance, the algorithm attempts to satisfy
this ``fuzzy constraint''.  We also calculate the number of encounters
between visitors.  After the itineraries are created, each visitor's campsite
listing is checked to determine how many visitors passed and were passed by
this specific visitor.  By combining the amount of erring from ideal travel distance
and encounters with other visitors, we obtain the aforementioned effect on
satisfaction.  This effect is calculated post-solution.
Throughout the course of operation, the solver relies heavily on several
parameters to relax the problem and make solutions more attainable.

\subsection{Usage of High-Level Parameters}
\label{sec:high-params}
% visitor size
% river length (maybe)
% days in the season
% number of sites
Given some of the information in the prompt and our findings, as discussed
in \textbf{Sections \ref{sec:defs}} and \textbf{\ref{sec:assumptions}}, our program
is designed to adjust its operation depending on various conditions.  The
parameter of greatest variability, number of  visitors, effects how
many visitors we generate at the beginning, the load distribution on portions
of the season, and some of the satisfaction scores for other visitors. Varying
this value was crucial in analyzing the model, discussed in \textbf{Section
\ref{sec:results}}.

Values that tend to be static include the river length, number of days in
the season, and number of campsites.  While each of these is capable of
drastically effecting the outcome of any given attempt, the values have a
well-defined range (see \textbf{Section \ref{sec:assumptions}}).  The problem
could be generalized beyond these variables such that the CSP solver attempts
to find optimal values for these as well, but this would likely place our
problem into the realm of truly NP-hard problems.


\subsubsection{Tolerance}
Built into the algorithm is a $tolerance$ variable: designed to create a
range for the distance a  visitor may cross in a given day.  This variable
is straightforward at the outset but provides several neat nuances for additional
complexity in the solution.  The introduction of a range to the CSP allows
for a larger domain of solutions and, therefore, variability.  In our case,
we desire such flexibility, to account for the real-world implications of
directing human beings.  Tolerance extends within the model as a sort of
aggregate for many of the various instances where we might expect variability
--- such as inclement weather, cancellations, or slow  visitors---,
without having to introduce variability into every aspect.  This may seem
like a stretch of the model's design, but any fuzziness for
any aspect introduces variability into the solutions.  We are satisfied that
tolerance is an opportunistic variable for allowing more visitors to be
scheduled and thus for the rafting season to be enjoyed by more individuals.



\section{Predictions and Analysis}
% address all of our graphs
% address the question of how to mix the visitors (i.e. oar versus motorized)
\subsection{General Expectations}
\label{sec:expectations}
Before creating the model we developed a few “common sense” expectations of
the results. For instance, it makes sense that increasing the visitor size
would make it harder to schedule each visitor at their desired time. The more
visitors one is trying to schedule, the faster the schedule fills up. However,
as the number of visitors on the river during a season increases, it would
become more and more likely that visitors would pass each other. Passing
would even be mandatory if the visitors are travelling at different daily
speeds; consider a visitor with a daily speed of 8 mph leaving a few days
after a visitor travelling at 4 mph.

We also expected that as the tolerance for which campsite visitors stayed at
each night increased, so would the ease of scheduling each visitor. The
greater the tolerance, the greater the likelihood of finding a site to stay
at if other visitors are also in the area. It also “made sense” to us, however,
that an increase in tolerance would cause a decrease in overall happiness.
An increase in tolerance would allow visitors to travel less than or more
than their desired time each day. This would decrease their happiness.

It was expected that assigning more visitors to motor boats (those that travel
at 8mph) would increase the success rate. Visitors that travel faster would
spend fewer days on the river, freeing up space for other campers faster,
and allowing more visitors to travel on the river each season. We also
predicted that seasons with more uniform visitor speeds (e.g. seasons where
all visitors travel at 4mph, or where all visitors travel at 8mph) would be
happier. This is because visitors travelling at the same speed are less likely
to pass each other, or to want to stay at the same campsite on the same night.
The greater the variation in visitor speed, the more likely it is faster
visitors would have to pass slower visitors.

\subsection{Results}
\label{sec:results}
Generally speaking, the results of the model were in line with our
expectations. For seasons with a 2-site camping tolerance, the success rate
was seen to decrease as the number of visitors increased
(\textbf{Figure~\ref{fig:sparklines} Spark 1}).

\begin{figure}[b]
  \centering
  \includegraphics[scale=.9]{imgs/sparklines.png}
  \caption{1: Success Rate vs. Visitors 2: Satisfaction
    vs. Visitors 3: Satisfaction vs. Visitors
    4: Tolerange vs. Success Rate and Satisfaction 5: Tolerance vs. Success Rate and Satisfaction}
  \label{fig:sparklines}
\end{figure}

The success rate was one-hundred percent up for seasons with 5, 50, and 100
visitors. Once the number of visitors per season increased past 100, the
success rate began decreasing at an increasing rate. The success rate
dropped from 100 percent to 91 percent between 100 and 300 campers, then
from 91 percent to 25 percent between 300 and 500 campers. For seasons with
more than 500 campers, the success rate leveled off, appearing to have an
asymptote just above zero percent. The success rate for a season with 600
visitors was five percent.

Similarly, the average happiness of each of the visitors decreased as the
number of visitors increased(\textbf{Figure~\ref{fig:sparklines} Spark 2}).
Once the number of visitors per season increased past 100, the
happiness rate began decreasing at an increasing rate. The happiness dropped
from 97.9 percent to 89.4 percent between 100 and 300 campers, then from 89.4
percent to 24.5 percent between 300 and 500 campers. For seasons with more
than 500 campers, the happiness rate leveled off, appearing to have an
asymptote just above zero percent. The happiness for a season with 600 visitors
was 4.9 percent.

For seasons with a  1-site tolerance, the effect of visitor size was seen to
increase(\textbf{Figure~\ref{fig:sparklines} Spark 3}). Both the
success rate and happiness of visitors began dropping with a visitor size of
only 50 people. The success rate dropped from 100 percent to 96 percent
when visitor sized increased from 5 to 50 visitors per season. By just 250
visitors, both the success rate and the happiness were zero percent.

The tolerance for which campsite visitors stayed at each night had a positive
effect on success rate of the scheduler and happiness of the visitors
(\textbf{Figure~\ref{fig:sparklines} Spark 4}).  For seasons with
200 visitors, a tolerance of zero campsites had a success rate of 0 percent
and happiness of zero percent. Increasing the tolerance to one campsite
resulted in a success rate of 52 percent and a happiness of 4.9 percent.
Increasing the tolerance to two campsites had the greatest effect, raising
success rate to 97.8 percent and happiness to 95.9 percent. With any greater
tolerances, the success rate was one-hundred percent, and the happiness
leveled off at 97.8 percent.

For seasons with 500 visitors, a greater tolerance was needed to achieve
similar results (\textbf{Figure~\ref{fig:sparklines} Spark 5}).
A tolerance of 0 campsites
had a success rate of 0; a tolerance of 1 campsite had a success rate of
only 2.6 percent; 2 campsites, 28 percent; 3 campsites, 90 percent. The
success rate was only 100 percent with a tolerance of greater than 4 campsites,
and the happiness was 98 percent.

Assigning more visitors to motorboats (those that travel at 8 mph) was seen
to increase the success rate. For a homogeneous season of 8 mph visitors, the
success rate was 98.2 percent and the happiness 74 percent. At the number
of 4 mph boats increased, both success rate and happiness steadily decreased.
With half rowboats and half motorboats, the success rate was 92 percent and
the happiness 72 percent. With a 9:1 ratio of rowboats to motorboats, the
success rate of the season was only 89.9 percent. However, there is a jump
in success rate to 99 percent once all visitors are travelling at 4 mph. This
jump in successes is not accompanied by a jump in happiness; with all
rowboats, the happiness is only 70 percent.

\subsection{Analysis}
\label{sec:capacity}
As defined in our definitions section, CARRYING CAPACITY IS []. With a
tolerance of 2 campsites (\textbf{Figure~\ref{fig:visitors-t2}}), we found the river could manage 100 visitors per
season with a 100 percent success rate and a happiness of at least 97 percent.
It is worth noting, however, that doubling the capacity to 200 visitors per
seasons yields a success rate of 99.5 percent, and a happiness still greater
than 97 percent. Once the number of visitors per season increases past 200,
both success rate and tolerance begin to drop rapidly.
\begin{figure}[h]
  \centering
  \includegraphics[scale=.6]{imgs/Graph_VistorsChanges_Tolerance2.png}
  \caption{Groups per Season vs. Success Rate and Satisfaction with Tolerance 2}
  \label{fig:visitors-t2}
\end{figure}
With a tolerance of 1 campsite, we found the river could only manage about
five visitors per season with a success rating of 100 percent. The average
happiness of these visitors is 97.7 percent. After the number of visitors per
season increases above 50, both happiness and success rate begin dropping
rapidly.
\begin{figure}[h]
  \centering
  \includegraphics[scale=.6]{imgs/Graph_VistorsChanges_Tolerance1.png}
  \caption{Groups per Season vs. Success Rate and Satisfaction with Tolerance 1}
  \label{fig:visitors-t1}
\end{figure}

OUR CARRYING CAPACITY - X IS THE ANSWER


SANITY CHECK USING NUMBER OF PEOPLE EVERY YEAR AND SUCH

\section{Final Thoughts}
\label{sec:conclusions}
\subsection{Advantages to the Approach}
\label{sec:pros}
Throughout, we have seen that a constraint satisfaction approach to scheduling
visitors to a river is a natural fit.  This approach allows for a logical
approach to the scheduling process, and for fairly quick solutions.  In most
cases, solutions were found on mid-grade hardware in less than ten minutes.
We also found that the structure of our solution in both code and at a
higher level lent itself well to the adjustment of any number of parameters.
Moreover, there exists a great deal of flexibility once one begins to define
``fuzzy values'' that we can assign to the variables.  Given more time,
extending the model with these fuzzy values would likely lead to better
assignments and even more interesting results.

\subsection{Detractors from the Approach}
\label{sec:cons}
river variation
perfection of visitors
vary start date
In our assumptions (\textbf{Section \ref{sec:assumptions}}) we note that the
visitors are basically perfect.  This, of course, is not true.  Visitors will
probably vaguely follow the schedule given them: due either to inclement
weather, rudeness, or physical constraints.  It is not immediately apparent
how to deal with this shortcoming other than simply accepting that not all
variables can be accounted for.  Padding the schedules with ``free days'' on
which any amount of travel lag may be recuperated.

The use of a consistent river in our model is less detrimental to the
overall performance and accuracy.  A realistic river would include areas
where passing other groups is impossible, visitors travel faster or slower,
and the potential for divisions of the river.  These could all be built in
as constraints on the campsites and their assignment to visitors.  Should
we have more time, this would likely be incorporated into the model.


\subsection{Future Extensions}
\label{sec:extensions}
Given more time there are several additions that would improve the technical
and practical performance of the model:
\begin{itemize}
\item The addition of a filtering technique to detect early failure in the
backtracking search.  This is often performed as what is called
``arc-consistency''.  Regretfully, it is difficult to implement for our
n-ary constraints.
\item Utilizing a fuzzy constraint for departure days would allow groups
to have a window in which they might leave.  This is likely to increase the
number of possible visitors, considering that it is nearly a direct analog
to tolerance.
\item The schedule might be modified to include days of rest for campsites
on a rotating basis.  This would not aid in adding more visitors (in fact it
would decrease the success rate), but it would assist in land management
practices and encourage a healthy use of the river ecosystem.
\end{itemize}


\subsection{Final Recommendations}
\label{sec:final}
Designating an exact number of visitors to raft down a river each year is
a complex problem that may not have only one solution. Managers will need
to be assertive in designating a tolerance, allotting departure days and
boat types to visitors, and instructing visitors to follow their schedules
carefully. Given certain parameters, however, we recommend a tolerance of
2 campsites, and about 300 visitors per season. A tolerance of 2 campsites
permits a large number of visitors per season while maintaining acceptable
visitor satisfaction. Any greater tolerance would create a sharp decrease
in satisfaction as visitors as forced to travel unfortunate distances; any
smaller tolerance greatly decreases the chance of a successful schedule for
any large number of visitors. Any more than 300 visitors is likely to drop
both the chance of a successful season and the happiness of any visitors.


\newpage


\bibliographystyle{amsplain}
\bibliography{mcm-2012-paper.bib}

\end{document}
